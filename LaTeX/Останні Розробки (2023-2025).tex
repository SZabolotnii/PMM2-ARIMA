\subsection{Останні Розробки (2023-2025)}
\label{subsec:recent_developments}

Останні три роки ознаменувалися значним прогресом у методології моделювання часових рядів з негаусовими характеристиками. Нові підходи можна класифікувати за трьома основними напрямками: глибинне навчання, адаптивні методи з урахуванням режимних зрушень, та робастне оцінювання за структурних зламів.

\subsubsection{Глибинне Навчання для Часових Рядів}

Методи глибинного навчання демонструють вражаючі результати у прогнозуванні складних часових рядів, особливо для нелінійних та високовимірних даних. Однак їхнє застосування до проблеми негаусових інновацій має специфічні особливості.

\paragraph{Архітектури на основі Transformers.}

Останні дослідження показують, що архітектури Transformer, адаптовані для часових рядів, можуть неявно моделювати негаусові характеристики через механізм уваги (attention mechanism). Проте ці моделі є:
\begin{itemize}
    \item \textbf{Data-hungry}: потребують тисяч спостережень для стабільного навчання, тоді як класичні ARIMA працюють з $N \geq 100$
    \item \textbf{Black-box}: відсутність інтерпретованих параметрів ускладнює економетричну інференцію
    \item \textbf{Computationally intensive}: навчання вимагає GPU та займає години, порівняно з секундами для PMM2
\end{itemize}

Важливо відзначити, що глибинні моделі оптимізовані для мінімізації prediction error, а не для ефективного \textit{оцінювання параметрів} основного процесу генерації даних. PMM2, навпаки, надає консистентні оцінки структурних параметрів $\boldsymbol{\theta} = (\phi_1, \ldots, \phi_p, \theta_1, \ldots, \theta_q)$ з явною асимптотичною теорією.

\paragraph{Гібридні ARIMA-Neural підходи.}

Dowe et al. (2025)~\cite{dowe2025novel} запропонували гібридну ARFIMA-ANN архітектуру, де ARFIMA моделює довгу пам'ять, а нейронна мережа -- нелінійні залишки. Їхні результати на фінансових та екологічних даних демонструють покращення RMSE на 8--15\% порівняно з чистим ARFIMA. Однак автори використовують MLE для початкового ARFIMA, що призводить до втрати ефективності за негаусових інновацій -- саме тієї проблеми, яку PMM2 вирішує.

\textbf{Синергія з PMM2}: Потенційна інтеграція полягає у використанні PMM2 для ефективного оцінювання лінійної компоненти, а нейронних мереж -- для моделювання нелінійних залишків:
\begin{align}
    \text{Stage 1:} \quad & \hat{\boldsymbol{\theta}}_{\text{PMM2}} \text{ оцінює ARIMA параметри} \label{eq:hybrid_stage1} \\
    \text{Stage 2:} \quad & \text{NN моделює} \; \hat{\varepsilon}_t - f_{\text{NN}}(\hat{\varepsilon}_{t-1}, \ldots, \hat{\varepsilon}_{t-L}) \label{eq:hybrid_stage2}
\end{align}

Така комбінація зберігає інтерпретованість лінійної частини (через PMM2) та гнучкість для нелінійних патернів (через NN).

\subsubsection{Адаптивні ARIMA з Режимними Зрушеннями}

Багато фінансових та економічних часових рядів демонструють зміни режиму (regime switching) -- зсуви у параметрах або волатильності через зовнішні шоки (фінансові кризи, пандемії, політичні події). Класичні ARIMA припускають стабільність параметрів $\boldsymbol{\theta}$, що порушується за умов структурних зламів.

\paragraph{Markov-Switching ARIMA.}

Сучасні підходи інтегрують ARIMA з Markov-Switching моделями, де параметри залежать від латентного стану $s_t \in \{1, \ldots, K\}$:
\begin{equation}
\label{eq:ms_arima}
\Phi_{s_t}(B) \Delta^d y_t = \Theta_{s_t}(B) \varepsilon_t, \quad \varepsilon_t \sim F_{s_t}(\cdot)
\end{equation}

Hamilton (1989) запровадив цей підхід для AR моделей, але оцінювання через EM алгоритм із гаусовим припущенням втрачає ефективність, коли $F_{s_t}$ є негаусовим у кожному режимі.

\textbf{PMM2 у MS контексті}: Природне розширення -- застосувати PMM2 для оцінювання $\boldsymbol{\theta}_{s}$ окремо в кожному ідентифікованому режимі:
\begin{enumerate}
    \item \textbf{Крок 1}: Ідентифікувати режими через Viterbi algorithm або Bayesian filtering
    \item \textbf{Крок 2}: Для кожного режиму $s$, де $n_s \geq 200$ спостережень, застосувати PMM2 до підвибірки $\{y_t : s_t = s\}$
    \item \textbf{Крок 3}: Порівняти кумулянти $(\gamma_{3,s}, \gamma_{4,s})$ між режимами -- якщо вони суттєво відрізняються, PMM2 забезпечить різні рівні ефективності
\end{enumerate}

Це дозволяє адаптувати метод оцінювання до режимно-специфічних характеристик інновацій.

\paragraph{Time-Varying Parameter ARIMA.}

Альтернативою дискретним режимам є моделі з плавно змінюваними параметрами (TVP-ARIMA), де $\phi_{i,t}$ еволюціонує згідно з випадковим блуканням або іншим стохастичним процесом. Байєсовські підходи (Particle Filter, Kalman Filter з non-Gaussian innovations) є стандартом, але страждають від computational burden.

PMM2 може бути адаптований до TVP через \textit{rolling window} або \textit{recursive} оцінювання:
\begin{itemize}
    \item \textbf{Rolling Window PMM2}: Оцінювати $\hat{\boldsymbol{\theta}}_t$ на вікні $[t-W, t]$ довжини $W$, re-compute кумулянти на кожному кроці
    \item \textbf{Exponentially Weighted PMM2}: Використати експоненційно зважені кумулянти $\tilde{\mu}_k(t) = \sum_{i=1}^{t} \lambda^{t-i} \varepsilon_i^k$ з коефіцієнтом забування $\lambda \in (0.95, 0.99)$
\end{itemize}

Такі адаптивні версії дозволяють відстежувати зміни у розподілі інновацій в реальному часі, що критично важливо для фінансових застосувань.

\subsubsection{Робастне Оцінювання за Структурних Зламів}

Структурні злами (structural breaks) -- різкі, одноразові зсуви у середньому, тренді або параметрах ARIMA -- є поширеним явищем у економічних даних (зміна монетарної політики, приєднання до валютного союзу, технологічні революції).

\paragraph{Tests for Structural Breaks.}

Класичні тести (Chow test, CUSUM, Bai-Perron sequential testing) дозволяють ідентифікувати точки зламу $t_1, \ldots, t_M$. Після їх виявлення, стандартний підхід -- оцінити окремі ARIMA моделі на кожному сегменті $[t_m, t_{m+1}]$.

\textbf{PMM2 advantage}: Якщо злам впливає на розподіл інновацій (наприклад, зміна з симетричного на асиметричний режим через кризу), PMM2 автоматично адаптується через re-estimation кумулянтів на кожному сегменті. Класичні методи з гаусовим припущенням не виявляють такої зміни.

\paragraph{Robust Estimation in Presence of Breaks.}

Reisen et al. (2024)~\cite{reisen2024robust} запропонували M-Whittle estimator, що є робастним до викидів через обмежену influence function. Однак цей метод орієнтований на симетричні важкі хвости (Student-t), а не на асиметрію.

Dinamarca et al. (2025)~\cite{dinamarca2025modeling} розробили SARIMAX з skew-normal innovations для метеорологічних даних з асиметричними залишками. Їхній підхід вимагає повної специфікації розподілу (skew-normal density), тоді як PMM2 використовує лише перші чотири моменти.

\textbf{Порівняльні переваги}:
\begin{table}[h]
\centering
\caption{Порівняння сучасних методів для негаусових ARIMA}
\label{tab:method_comparison_recent}
\begin{tabular}{@{}lcccc@{}}
\toprule
\textbf{Метод} & \textbf{Асиметрія} & \textbf{Breaks} & \textbf{Інтерпретованість} & \textbf{Обчислення} \\
\midrule
Deep Learning (LSTM) & Implicit & Adaptive & Low & High \\
Hybrid ARFIMA-ANN & Implicit & No & Medium & High \\
MS-ARIMA (Gaussian) & No & Yes & High & Medium \\
TVP-ARIMA (Bayesian) & Possible & Yes & Medium & Very High \\
M-Whittle (Robust) & No & Partial & High & Medium \\
SARIMAX-SkewNormal & Yes & No & High & Medium \\
\rowcolor{green!20}
\textbf{PMM2-ARIMA} & \textbf{Yes} & \textbf{Yes (segmented)} & \textbf{High} & \textbf{Low} \\
\bottomrule
\end{tabular}
\end{table}

\paragraph{PMM2 у контексті структурних зламів.}

Для часових рядів з підозрою на злам, рекомендується наступна процедура:
\begin{enumerate}
    \item \textbf{Тест на злам}: Bai-Perron test для ідентифікації $\hat{t}_1, \ldots, \hat{t}_M$
    \item \textbf{Сегментація}: Розділити ряд на $M+1$ сегментів
    \item \textbf{PMM2 per segment}: Для кожного сегменту $m$ з $n_m \geq 150$ спостережень:
    \begin{itemize}
        \item Оцінити $\hat{\boldsymbol{\theta}}_m$ через PMM2
        \item Обчислити $\hat{\gamma}_{3,m}$, $\hat{\gamma}_{4,m}$
        \item Якщо $|\hat{\gamma}_{3,m}| < 0.5$, перейти до OLS для цього сегменту
    \end{itemize}
    \item \textbf{Діагностика}: Перевірити стабільність $\hat{\boldsymbol{\theta}}_m$ між сегментами -- значні зміни вказують на справжній структурний злам
\end{enumerate}

Така стратегія поєднує переваги тестів на злам (виявлення точок зміни) з адаптивністю PMM2 (ефективне оцінювання за різних розподілів).

\subsubsection{Синтез: Позиціонування PMM2}

Огляд останніх розробок демонструє, що PMM2 займає унікальну нішу в екосистемі методів для негаусових ARIMA:

\begin{itemize}
    \item \textbf{Проти Deep Learning}: PMM2 забезпечує інтерпретовані параметри, швидкі обчислення ($<$ 1 сек), та працює з малими вибірками ($N \geq 200$). DL переважає для складних нелінійних патернів та великих даних ($N > 10{,}000$).

    \item \textbf{Проти MS/TVP моделей}: PMM2 є простішим та обчислювально дешевшим для стаціонарних періодів. Для даних з явними режимними зрушеннями, комбінація MS-framework + PMM2 per regime є перспективним напрямком.

    \item \textbf{Проти робастних методів}: PMM2 оптимальний для \textit{асиметричних} розподілів без викидів (Gamma, Lognormal, Chi-squared). M-estimators кращі для \textit{симетричних} важких хвостів з викидами (Student-t, Cauchy).

    \item \textbf{Проти parametric skewed specifications}: PMM2 не потребує вибору конкретного розподілу (skew-normal, skew-t), використовуючи лише кумулянти. Це робить метод більш гнучким, але потенційно менш ефективним, якщо справжній розподіл відомий.
\end{itemize}

\textbf{Висновок}: PMM2 є ефективним, інтерпретованим та обчислювально легким рішенням для статичних ARIMA моделей з асиметричними інноваціями. Для динамічних параметрів (TVP), режимних зрушень (MS), або складних нелінійностей (DL), доцільна інтеграція PMM2 як компоненти в більш складні фреймворки.